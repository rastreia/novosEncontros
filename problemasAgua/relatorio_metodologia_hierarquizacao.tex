\documentclass[a4paper, 12pt, openright, oneside, english, brazil, article]{abntex2}
\usepackage[brazil]{babel}
\usepackage{graphicx}
\usepackage[utf8]{inputenc}
\usepackage{wrapfig}
\usepackage{lscape}
\usepackage{rotating}
\usepackage{epstopdf}
\usepackage[alf]{abntex2cite}
\usepackage[a4paper, left=3cm, right=2cm, top=3cm, bottom=2cm]{geometry}
\usepackage{indentfirst}
\usepackage{longtable}
\usepackage{amsmath}
\usepackage{verbatim}
\usepackage{TikZ}
\pagestyle{plain}

\titulo{Nota técnica - Relatório metodológico sobre a seleção de domicílios beneficiados pelo programa Sementes Presentes}


\begin{document}
		
	\thispagestyle{empty}
	\begin{tikzpicture}[remember picture, overlay]
	\node[inner sep=0pt] at (current page.center) {%
		\includegraphics[width=\paperwidth,height=\paperheight]{Capa_rastreia.png}%
	};
	\end{tikzpicture}
	\cleardoublepage{}
	
	
	\newpage
	
	\pretextual
	\maketitle
	
	\textual
	
	\section{Introdução}
	
	O programa \textbf{Novos Encontros} é uma política do Governo do Estado de Minas Gerais gerida pela Diretoria de Programas de Enfrentamento à Pobreza no Campo. Dentro dele, foram desenvolvidas duas ações intituladas ``Brasil sem Miséria'' e ``Sementes Presentes''. O primeiro consiste, de maneira geral, de concessão de bolsas de auxílio produtivo e treinamento para a produção agrícola. O segundo conssite, em linhas gerais, de distribuição de kits de sementes e irrigação.
	
	O programa \textbf{Novos Encontros} converge com umas das 6 principais metas de governo da atual gestão, qual seja, o enfrentamento à pobreza no campo. É senso comum entre os estudiosos da pobreza que esta não se manifesta de uma forma simples ou convergente mas de forma difusa e multidimensional \ldots...
	
	Ação intersetorial \ldots
	
	Convergência técnica entre temas \ldots
	
		
	\section{Metodologia}
	
	O estabelecimento de critérios de prioridade para a seleção de domicílios beneficiados pelas ações do programa \textbf{Novos Encontros}, como vimos não é tarefa trivial. Para realizar essa tarefa faz-se necessário recorrer a três fontes de informação, a saber, (1) os critérios dados \textit{a priori} pela diretoria responsável pela gestão das ações, (2) o embasamento teórico vigente acerca da pobreza multidimensional e (3) modelos estatísticos que permitam identificar na realidade imediata de atuação o impacto de variáveis de interesse sobre a condição de pobreza.
	
	Os critérios dados \textit{a priori} para hierarquização de domicílios a receberem o auxílio da ação ``Brasil sem Miséria'' são:
	
	\begin{enumerate}
		\item Domicílios que possuam renda mensal até R\$85,00 \textit{per capita};
		\item Não ter recebido Bolsa Família e
		\item Domicílios rurais.
	\end{enumerate}

	Apesar de ser uma área de pesquisa profícua, os estudos sobre pobreza multidimensional infelizmente possuem pouca informação empírica que sirva de base para a tomada de decisão política. Desse modo, apresentou-se a possibilidade da elaboração de um modelo estatístico que permitisse verificar dentro do espectro dos beneficiários de programas sociais no Estado de MG, os efeitos de algumas variáveis de interesse sobre a pobreza. Essa análise não visa fornecer dados para investigações gerais ou generalizáveis acerca da pobreza no campo mas orientar a escolha de beneficiários específicos. Desse modo, torna-se interessante a escolha do CADASTRO ÚNICO como base de dados para a elaboração do modelo estatístico. O CADASTRO ÚNICO ``é um conjunto de informações sobre as famílias brasileiras em situação de pobreza e extrema pobreza\footnote{Fonte: Site da Caixa Econômica Federal. Disponível em \url{http://www.caixa.gov.br/cadastros/cadastro-unico/Paginas/default.aspx}, acesso em 08 de junho de 2017.}'' utilizado pelo Governo Federal, Estados e Municípios para a implementação de políticas sociais. Nele são cadastradas famílias que ganham até meio salário mínimo \textit{per capita} ou famílias que ganham até três salários mínimos de renda mensal total. Esse banco de dados representa, portanto, uma parcela da população já pobre. Um modelo estatístico elaborado a partir desses dados nos aponta dentro desse espectro específico, variáveis com impacto nas chances de pobreza. Dito de outra forma, queremos investigar dentro de uma parcela de pessoas pobres, o que faz com que aumentem as chances de cada família estar ou não em situação de extrema pobreza. Para realizar tal feito, escolhemos trabalhar com o modelo logístico hierárquico.
	
	\subsection{O modelo hierárquico}
	
	O modelo logístico assume uma variável resposta binária e estima chances de sucesso e fracasso para essa variável dadas as demais variáveis de controle inseridas no modelo. Para os fins de nossa análise, escolhemos trabalhar com uma sofisticação a mais, qual seja, a inserção de um intercepto aleatório no modelo, ou seja, a utilização de um modelo hierárquico. Segundo Crepalde e Silveira (2016):
	
	\begin{citacao}
	Esses modelos surgiram nos estudos em Ciências da Educação considerando que as variáveis abordadas não deveriam ser vistas no mesmo plano, mas aninham-se de forma hierárquica em níveis diferentes de análise.\footnote{CREPALDE, Neylson João Batista Filho; SILVEIRA, Leonardo Souza. Desempenho universitário no Brasil: estudo sobre desigualdade educacional com dados do Enade 2014. Revista Brasileira de Sociologia-RBS, v. 4, n. 7, 2016. Disponível em \url{http://www.sbsociologia.com.br/revista/index.php/RBS/article/view/155}, acesso em 8 de junho de 2017.}		
	\end{citacao}
	
	Esse modelo possui a vantagem de ``contemplar as correlações e a variação dos coeficientes estimados entre os grupos aninhados, o que escapa aos modelos com apenas um nível\footnote{\textit{ibid.}}''.

	\ldots
	
	
	
% FINAL
\newpage

\thispagestyle{empty}
\begin{tikzpicture}[remember picture, overlay]
\node[inner sep=0pt] at (current page.center) {%
	\includegraphics[width=\paperwidth,height=\paperheight]{Contracapa_rastreia.png}%
};
\end{tikzpicture}
\cleardoublepage{}

\end{document}
	
	
