\documentclass[a4paper, 12pt, openright, oneside, english, brazil, article]{abntex2}
\usepackage[brazil]{babel}
\usepackage{graphicx}
\usepackage[utf8]{inputenc}
\usepackage{wrapfig}
\usepackage{lscape}
\usepackage{rotating}
\usepackage{epstopdf}
\usepackage[alf]{abntex2cite}
\usepackage[a4paper, left=3cm, right=2cm, top=3cm, bottom=2cm]{geometry}
\usepackage{indentfirst}
\usepackage{longtable}
\usepackage{amsmath}
\usepackage{verbatim}
\usepackage{TikZ}
\pagestyle{plain}

\titulo{Nota técnica - Análise dos problemas territoriais do acesso à água}


\begin{document}
		
	\thispagestyle{empty}
	\begin{tikzpicture}[remember picture, overlay]
	\node[inner sep=0pt] at (current page.center) {%
		\includegraphics[width=\paperwidth,height=\paperheight]{Capa_rastreia.png}%
	};
	\end{tikzpicture}
	\cleardoublepage{}
	
	
	\newpage
	
	\pretextual
	\maketitle
	
	\textual
	
	\section{Introdução}
	
	Como estratégia de ação, o programa NOVOS ENCONTROS se propõe a analisar ações de acesso a água realizadas pelos diversos órgãos do Estado ou em parceria com o mesmo objetivando intervenções que aumentem significativamente a eficiência dos processos e otimizem os resultados obtidos. A partir de dados fornecidos pelos órgãos, o RASTREIA (Diretoria de Monitoramento e Avaliação de Programas Especiais) realizou uma análise sobre os principais problemas apontados buscando nortear as decisões do NOVOS ENCONTROS.
	
	\section{Tecnologias utilizadas}
	
	No gráfico \ref{tec} e na tabela \ref{tectab} apresentamos a distribuição das tecnologias em uso pelos órgãos responsáveis.
	
	\begin{figure}[!h]
		%\centering
		\caption{Tecnologias}
		\label{tec}
		%\includegraphics[scale=.6]{tecnologias.png}
		\includegraphics[scale=.8]{tecnologias_orgaos.png}
	\end{figure}

%	\begin{figure}[!h]
%		\centering
%		\caption{Tecnologias por órgão}
%		\label{tec-orgao}
%		\includegraphics[scale=.6]{tecnologias_orgaos.png}
%	\end{figure}

	% latex table generated in R 3.3.3 by xtable 1.8-2 package
	% Thu Apr 20 13:38:38 2017
	\begin{scriptsize}
	\begin{longtable}{rrr}
		\caption{Tecnologias} \\ 
		\hline
		& Frequência & Percentual \\ 
		\hline
		Barragem & 5.00 & 1.33 \\ 
		Implantação de Sistemas de Dessalinização  & 26.00 & 6.90 \\ 
		Poço & 345.00 & 91.51 \\ 
		Sistema de abastecimento de água- Reservatorio e distribuição & 1.00 & 0.27 \\ 
		\hline
		Total & 377.00 & 100.00 \\ 
		\hline
		\hline
		\label{tectab}
	\end{longtable}
	\end{scriptsize}


	
	É possível observar que \textbf{poços} estão presentes em 91,5\% das ações mapeadas. Entretanto, ao observarmos a distribuição de tecnologias por órgão responsável, podemos perceber que a SECIR é a única instituição responsável por \textbf{implantação de sistemas de dessalinização} e somente a COPANOR possui ações com barragens. Devido à predominância de tecnologias relacionadas a poços, priorizaremos nossas análises subsequentes nesses problemas.

	\section{Poços fora de operação por município}
	
	Preliminarmente, investigamos a quantidade de poços que não estão em funcionamento e sua distribuição nas cidades mineiras. Essa distribuição encontra-se na tabela \ref{poc-operacao}.
	
	% latex table generated in R 3.3.3 by xtable 1.8-2 package
	% Wed Apr 12 10:32:00 2017
	\begin{scriptsize}
	% latex table generated in R 3.3.3 by xtable 1.8-2 package
	% Thu Apr 20 13:59:45 2017
	\begin{longtable}{rrl}
		\caption{Poços fora de operação} \\ 
		\hline
		Frequência & \% Válido & nome \\ 
		\hline
		12.00 & 4.67 & SÃO FRANCISCO \\ 
		8.00 & 3.11 & Padre Carvalho \\ 
		7.00 & 2.72 & Mato Verde \\ 
		6.00 & 2.33 & TAIOBEIRAS \\ 
		6.00 &  & NA's \\ 
		5.00 & 1.95 & ARAÇUAÍ \\ 
		5.00 & 1.95 & Catuti \\ 
		5.00 & 1.95 & JAÍBA \\ 
		5.00 & 1.95 & JEQUITINHONHA \\ 
		5.00 & 1.95 & LAGOA DOS PATOS \\ 
		5.00 & 1.95 & MATIAS CARDOSO \\ 
		5.00 & 1.95 & PORTEIRINHA \\ 
		5.00 & 1.95 & RIACHO DOS MACHADOS \\ 
		5.00 & 1.95 & SERRANÓPOLIS DE MINAS \\ 
		4.00 & 1.56 & CARAÍ \\ 
		4.00 & 1.56 & Chapada do Norte \\ 
		4.00 & 1.56 & Itamarandiba \\ 
		4.00 & 1.56 & MATO VERDE \\ 
		4.00 & 1.56 & Minas Novas \\ 
		4.00 & 1.56 & MONTE AZUL \\ 
		4.00 & 1.56 & NINHEIRA \\ 
		4.00 & 1.56 & Rio do Prado \\ 
		4.00 & 1.56 & Teófilo Otoni \\ 
		4.00 & 1.56 & Turmalina \\ 
		3.00 & 1.17 & BERILO \\ 
		3.00 & 1.17 & Comercinho \\ 
		3.00 & 1.17 & FRUTA DE LEITE \\ 
		3.00 & 1.17 & Janaúba \\ 
		3.00 & 1.17 & JENIPAPO DE MINAS \\ 
		3.00 & 1.17 & JOAÍMA \\ 
		3.00 & 1.17 & JORDÂNIA \\ 
		3.00 & 1.17 & MATA VERDE \\ 
		3.00 & 1.17 & MINAS NOVAS \\ 
		3.00 & 1.17 & MIRABELA \\ 
		3.00 & 1.17 & Novorizonte \\ 
		3.00 & 1.17 & PAI PEDRO \\ 
		3.00 & 1.17 & Poté \\ 
		3.00 & 1.17 & SALINAS \\ 
		3.00 & 1.17 & SANTA CRUZ DE SALINAS \\ 
		3.00 & 1.17 & SANTA FÉ DE MINAS \\ 
		3.00 & 1.17 & SANTA MARIA DO SUAÇUÍ \\ 
		3.00 & 1.17 & SENADOR MODESTINO GONÇALVES \\ 
		2.00 & 0.78 & Bandeira \\ 
		2.00 & 0.78 & Botumirim \\ 
		2.00 & 0.78 & CACHOEIRA DE PAJEÚ \\ 
		2.00 & 0.78 & CORONEL MURTA \\ 
		2.00 & 0.78 & CURRAL DE DENTRO \\ 
		2.00 & 0.78 & Felisburgo \\ 
		2.00 & 0.78 & FRANCISCO BADARÓ \\ 
		2.00 & 0.78 & ICARAÍ DE MINAS \\ 
		2.00 & 0.78 & Itaipé \\ 
		2.00 & 0.78 & JACINTO \\ 
		2.00 & 0.78 & JURAMENTO \\ 
		2.00 & 0.78 & Leme do Prado \\ 
		2.00 & 0.78 & LEME DO PRADO \\ 
		2.00 & 0.78 & MEDINA \\ 
		2.00 & 0.78 & MONJOLOS \\ 
		2.00 & 0.78 & Padre Paraíso \\ 
		2.00 & 0.78 & PADRE PARAÍSO \\ 
		2.00 & 0.78 & Palmópolis \\ 
		2.00 & 0.78 & PALMÓPOLIS \\ 
		2.00 & 0.78 & PATIS \\ 
		2.00 & 0.78 & PEDRA AZUL \\ 
		2.00 & 0.78 & PONTO CHIQUE \\ 
		2.00 & 0.78 & PONTO DOS VOLANTES \\ 
		2.00 & 0.78 & RIACHINHO \\ 
		2.00 & 0.78 & RIO DO PRADO \\ 
		2.00 & 0.78 & Rio Pardo de Minas \\ 
		2.00 & 0.78 & RUBIM \\ 
		2.00 & 0.78 & SÃO JOÃO DO PARAÍSO \\ 
		2.00 & 0.78 & VARGEM GRANDE DO RIO PARDO \\ 
		2.00 & 0.78 & VÁRZEA DA PALMA \\ 
		1.00 & 0.39 & Araçuaí \\ 
		1.00 & 0.39 & ARICANDUVA \\ 
		1.00 & 0.39 & BANDEIRA \\ 
		1.00 & 0.39 & DIVISA ALEGRE \\ 
		1.00 & 0.39 & Divisópolis \\ 
		1.00 & 0.39 & ITAOBIM \\ 
		1.00 & 0.39 & JAPONVAR \\ 
		1.00 & 0.39 & Ladainha \\ 
		1.00 & 0.39 & Mantena \\ 
		1.00 & 0.39 & Monte Formoso \\ 
		1.00 & 0.39 & Nova Módica \\ 
		1.00 & 0.39 & Ouro Verde de Minas \\ 
		1.00 & 0.39 & Ponto dos Volantes \\ 
		\hline
		\hline
		\label{poc-operacao}
	\end{longtable}
	\end{scriptsize}
	
	Os principais motivos pelos quais os poços não estão em funcionamento são apresentados na tabela \ref{mot-nao-operacao} e na figura \ref{motivos-grafico}. É possível identificar que a maior causa do não funcionamento é a \textbf{ausência de energia elétrica} seguido imediatamente pela \textbf{ausência de análise da água do local}. Nas tabelas \ref{motivo1}, \ref{motivo3} e \ref{motivo4}, apresentamos a distribuição de cada um dos principais motivos encontrados por município. O único caso de poço não operando devido a água imprópria para o consumo foi encontrado no município de Aricanduva.
	
	
	% Motivos de não operação
	
	\begin{figure}[!h]
		\centering
		\caption{Motivos de não operação}
		\label{motivos-grafico}
		\includegraphics[scale=.8]{motivos_grafico.png}
	\end{figure}
	
	
	\begin{scriptsize}
		% latex table generated in R 3.3.3 by xtable 1.8-2 package
		% Wed Apr 12 11:18:14 2017
		\begin{longtable}{rrr}
			\caption{Motivos de não operação} \\ 
			\hline
			Motivo & Frequência & Percentual \\ 
			\hline
			Sem análise de água & 141.00 & 39.17 \\
			Água imprópria para consumo humano & 1.00 & 0.28 \\  
			Tubulação não entregue & 24.00 & 6.67 \\
			Sem energização& 194.00 & 53.89 \\
			\hline
			Total & 360.00 & 100.00 \\ 
			\hline
			\hline
			\label{mot-nao-operacao}
		\end{longtable}
	\end{scriptsize}
	
	
	
	% Municípios por motivo 1
	
	\begin{scriptsize}
	% latex table generated in R 3.3.3 by xtable 1.8-2 package
	% Thu Apr 20 14:20:47 2017
	\begin{longtable}{rrl}
		\caption{Municípios sem análise de água} \\ 
		\hline
		Frequência & Percentual & Município \\ 
		\hline
		12.00 & 8.51 & MONTES CLAROS \\ 
		12.00 & 8.51 & SÃO FRANCISCO \\ 
		8.00 & 5.67 & Padre Carvalho \\ 
		8.00 & 5.67 & TAIOBEIRAS \\ 
		5.00 & 3.55 & ARAÇUAÍ \\ 
		5.00 & 3.55 & LAGOA DOS PATOS \\ 
		5.00 & 3.55 & MATIAS CARDOSO \\ 
		5.00 & 3.55 & PORTEIRINHA \\ 
		5.00 & 3.55 & RIACHO DOS MACHADOS \\ 
		5.00 & 3.55 & SERRANÓPOLIS DE MINAS \\ 
		4.00 & 2.84 & CARAÍ \\ 
		4.00 & 2.84 & MATO VERDE \\ 
		4.00 & 2.84 & MONTE AZUL \\ 
		4.00 & 2.84 & NINHEIRA \\ 
		4.00 & 2.84 & SALINAS \\ 
		3.00 & 2.13 & MATA VERDE \\ 
		3.00 & 2.13 & MINAS NOVAS \\ 
		3.00 & 2.13 & MIRABELA \\ 
		3.00 & 2.13 & PAI PEDRO \\ 
		3.00 & 2.13 & SANTA CRUZ DE SALINAS \\ 
		3.00 & 2.13 & SANTA FÉ DE MINAS \\ 
		3.00 & 2.13 & SENADOR MODESTINO GONÇALVES \\ 
		2.00 & 1.42 & LEME DO PRADO \\ 
		2.00 & 1.42 & MEDINA \\ 
		2.00 & 1.42 & MONJOLOS \\ 
		2.00 & 1.42 & PADRE PARAÍSO \\ 
		2.00 & 1.42 & PALMÓPOLIS \\ 
		2.00 & 1.42 & PATIS \\ 
		2.00 & 1.42 & PEDRA AZUL \\ 
		2.00 & 1.42 & PONTO CHIQUE \\ 
		2.00 & 1.42 & PONTO DOS VOLANTES \\ 
		2.00 & 1.42 & RIACHINHO \\ 
		2.00 & 1.42 & RIO DO PRADO \\ 
		2.00 & 1.42 & RUBIM \\ 
		2.00 & 1.42 & SÃO JOÃO DO PARAÍSO \\ 
		2.00 & 1.42 & VARGEM GRANDE DO RIO PARDO \\ 
		2.00 & 1.42 & VÁRZEA DA PALMA \\ 
		\hline
		\hline
		\label{motivo1}
	\end{longtable}
	\end{scriptsize}
	
	

	% Motivo água imprópria para o consumo
	
%	\begin{scriptsize}
%		% latex table generated in R 3.3.3 by xtable 1.8-2 package
%		% Wed Apr 12 11:35:42 2017
%		\begin{longtable}{rrl}
%			\caption{Municípios com poços com água imprópria para o consumo} \\ 
%			\hline
%			Frequência & Percentual & Município \\ 
%			\hline
%			5.00 & 71.43 & Cristália \\ 
%			2.00 & 28.57 & ARICANDUVA \\ 
%			\hline
%			\hline
%			\label{motivo2}
%		\end{longtable}
%	\end{scriptsize}


	% Motivo Tubulação não entregue
	
	\begin{scriptsize}
	% latex table generated in R 3.3.3 by xtable 1.8-2 package
	% Thu Apr 20 14:24:39 2017
	\begin{longtable}{rrl}
		\caption{Municípios com poços com tubulação não entregue} \\ 
		\hline
		Frequência & Percentual & Município \\ 
		\hline
		5.00 & 20.83 & RIACHO DOS MACHADOS \\ 
		5.00 & 20.83 & SERRANÓPOLIS DE MINAS \\ 
		3.00 & 12.50 & SANTA FÉ DE MINAS \\ 
		3.00 & 12.50 & SANTA MARIA DO SUAÇUÍ \\ 
		2.00 & 8.33 & JURAMENTO \\ 
		2.00 & 8.33 & RIO DO PRADO \\ 
		2.00 & 8.33 & SALINAS \\ 
		2.00 & 8.33 & VÁRZEA DA PALMA \\ 
		\hline
		\hline
		\label{motivo3}
	\end{longtable}
	\end{scriptsize}

	% Motivo Sem energização
	
	\begin{scriptsize}
	% latex table generated in R 3.3.3 by xtable 1.8-2 package
	% Thu Apr 20 14:25:38 2017
	\begin{longtable}{rrl}
		\caption{Municípios com poços sem energização} \\ 
		\hline
		Frequência & Percentual & Município \\ 
		\hline
		12.00 & 6.19 & MONTES CLAROS \\ 
		11.00 & 5.67 & SÃO FRANCISCO \\ 
		8.00 & 4.12 & Padre Carvalho \\ 
		6.00 & 3.09 & TAIOBEIRAS \\ 
		5.00 & 2.58 & Catuti \\ 
		5.00 & 2.58 & JAÍBA \\ 
		5.00 & 2.58 & LAGOA DOS PATOS \\ 
		5.00 & 2.58 & MATIAS CARDOSO \\ 
		5.00 & 2.58 & PORTEIRINHA \\ 
		5.00 & 2.58 & RIACHO DOS MACHADOS \\ 
		5.00 & 2.58 & SERRANÓPOLIS DE MINAS \\ 
		4.00 & 2.06 & ARAÇUAÍ \\ 
		4.00 & 2.06 & CARAÍ \\ 
		4.00 & 2.06 & MATO VERDE \\ 
		4.00 & 2.06 & MONTE AZUL \\ 
		4.00 & 2.06 & SALINAS \\ 
		3.00 & 1.55 & BERILO \\ 
		3.00 & 1.55 & Chapada do Norte \\ 
		3.00 & 1.55 & Comercinho \\ 
		3.00 & 1.55 & FRUTA DE LEITE \\ 
		3.00 & 1.55 & Itamarandiba \\ 
		3.00 & 1.55 & JENIPAPO DE MINAS \\ 
		3.00 & 1.55 & JEQUITINHONHA \\ 
		3.00 & 1.55 & JOAÍMA \\ 
		3.00 & 1.55 & MATA VERDE \\ 
		3.00 & 1.55 & MINAS NOVAS \\ 
		3.00 & 1.55 & MIRABELA \\ 
		3.00 & 1.55 & PAI PEDRO \\ 
		3.00 & 1.55 & SANTA CRUZ DE SALINAS \\ 
		3.00 & 1.55 & SANTA FÉ DE MINAS \\ 
		3.00 & 1.55 & SANTA MARIA DO SUAÇUÍ \\ 
		3.00 & 1.55 & SENADOR MODESTINO GONÇALVES \\ 
		2.00 & 1.03 & CACHOEIRA DE PAJEÚ \\ 
		2.00 & 1.03 & CORONEL MURTA \\ 
		2.00 & 1.03 & Felisburgo \\ 
		2.00 & 1.03 & FRANCISCO BADARÓ \\ 
		2.00 & 1.03 & ICARAÍ DE MINAS \\ 
		2.00 & 1.03 & JACINTO \\ 
		2.00 & 1.03 & JORDÂNIA \\ 
		2.00 & 1.03 & JURAMENTO \\ 
		2.00 & 1.03 & LEME DO PRADO \\ 
		2.00 & 1.03 & MEDINA \\ 
		2.00 & 1.03 & MONJOLOS \\ 
		2.00 & 1.03 & PADRE PARAÍSO \\ 
		2.00 & 1.03 & PALMÓPOLIS \\ 
		2.00 & 1.03 & PATIS \\ 
		2.00 & 1.03 & PEDRA AZUL \\ 
		2.00 & 1.03 & PONTO CHIQUE \\ 
		2.00 & 1.03 & PONTO DOS VOLANTES \\ 
		2.00 & 1.03 & RIACHINHO \\ 
		2.00 & 1.03 & RIO DO PRADO \\ 
		2.00 & 1.03 & RUBIM \\ 
		2.00 & 1.03 & SÃO JOÃO DO PARAÍSO \\ 
		2.00 & 1.03 & VARGEM GRANDE DO RIO PARDO \\ 
		2.00 & 1.03 & VÁRZEA DA PALMA \\ 
		1.00 & 0.52 & ARICANDUVA \\ 
		1.00 & 0.52 & Bandeira \\ 
		1.00 & 0.52 & BANDEIRA \\ 
		1.00 & 0.52 & CURRAL DE DENTRO \\ 
		1.00 & 0.52 & DIVISA ALEGRE \\ 
		1.00 & 0.52 & ITAOBIM \\ 
		1.00 & 0.52 & JAPONVAR \\ 
		1.00 & 0.52 & NINHEIRA \\ 
		\hline
		\hline
		\label{motivo4}
	\end{longtable}
	\end{scriptsize}

	
	\section{Problemas fundiários}
	Apenas 8\% do universo investigado apresenta dados sobre problemas fundiários. A grande quantidade de não respostas (\textit{missing values}) torna inviável a análise desta variável através dos dados disponibilizados.
	
	
	\section{Licença ambiental}
	
	\begin{figure}[!h]
		\centering
		\caption{Licença ambiental - possui outorga?}
		\label{outorga-fig}
		\includegraphics[scale=.8]{outorgas_orgaos.png}
	\end{figure}
	
	Com relação ao licenciamento ambiental, verificamos primeiro a distribuição de frequências das categorias da variável - ver tabela \ref{outorga} - bem como sua distribuição por órgão competente - ver figura \ref{outorga-fig}. Aqui constam apenas os dados fornecidos pelas instituições SEDINOR, SEDIR e IGAM. É possível verificar que aprox. 86\% das ações cadastradas válidas para análise (desconsiderando as ``não respostas'') não possuem outorga as quais estão concentradas com a SEDINOR. Na tabela \ref{semoutorga} apresentamos a distribuição das ações sem outorga por município.
	
	

	\begin{scriptsize}
	% latex table generated in R 3.3.3 by xtable 1.8-2 package
	% Thu Apr 20 14:34:35 2017
	\begin{longtable}{rrrr}
		\caption{Licença ambiental - possui outorga?} \\ 
		\hline
		& Frequência & Percentual & \% Válido \\ 
		\hline
		Não & 219.00 & 70.65 & 85.88 \\ 
		Outorgado pelo IGAM & 36.00 & 11.61 & 14.12 \\ 
		NA's & 55.00 & 17.74 &  \\ 
		Total & 310.00 & 100.00 & 100.00 \\ 
		\hline
		\hline
		\label{outorga}
	\end{longtable}
	\end{scriptsize}

	
	% Municípios sem outorga
	\begin{scriptsize}
	% latex table generated in R 3.3.3 by xtable 1.8-2 package
	% Thu Apr 20 14:42:28 2017
	\begin{longtable}{rrl}
		\caption{Municípios com poços sem outorga} \\ 
		\hline
		Frequência & Percentual & Município \\ 
		\hline
		12.00 & 5.48 & MONTES CLAROS \\ 
		12.00 & 5.48 & SÃO FRANCISCO \\ 
		8.00 & 3.65 & Padre Carvalho \\ 
		8.00 & 3.65 & TAIOBEIRAS \\ 
		5.00 & 2.28 & ARAÇUAÍ \\ 
		5.00 & 2.28 & Catuti \\ 
		5.00 & 2.28 & JAÍBA \\ 
		5.00 & 2.28 & JEQUITINHONHA \\ 
		5.00 & 2.28 & LAGOA DOS PATOS \\ 
		5.00 & 2.28 & MATIAS CARDOSO \\ 
		5.00 & 2.28 & MATO VERDE \\ 
		5.00 & 2.28 & PORTEIRINHA \\ 
		5.00 & 2.28 & RIACHO DOS MACHADOS \\ 
		5.00 & 2.28 & SALINAS \\ 
		5.00 & 2.28 & SERRANÓPOLIS DE MINAS \\ 
		4.00 & 1.83 & CARAÍ \\ 
		4.00 & 1.83 & Itamarandiba \\ 
		4.00 & 1.83 & MONTE AZUL \\ 
		4.00 & 1.83 & NINHEIRA \\ 
		3.00 & 1.37 & BERILO \\ 
		3.00 & 1.37 & Chapada do Norte \\ 
		3.00 & 1.37 & Comercinho \\ 
		3.00 & 1.37 & FRUTA DE LEITE \\ 
		3.00 & 1.37 & JAPONVAR \\ 
		3.00 & 1.37 & JENIPAPO DE MINAS \\ 
		3.00 & 1.37 & JOAÍMA \\ 
		3.00 & 1.37 & JORDÂNIA \\ 
		3.00 & 1.37 & LEME DO PRADO \\ 
		3.00 & 1.37 & MATA VERDE \\ 
		3.00 & 1.37 & MEDINA \\ 
		3.00 & 1.37 & MINAS NOVAS \\ 
		3.00 & 1.37 & MIRABELA \\ 
		3.00 & 1.37 & PAI PEDRO \\ 
		3.00 & 1.37 & PEDRA AZUL \\ 
		3.00 & 1.37 & SANTA CRUZ DE SALINAS \\ 
		3.00 & 1.37 & SANTA FÉ DE MINAS \\ 
		3.00 & 1.37 & SANTA MARIA DO SUAÇUÍ \\ 
		3.00 & 1.37 & SENADOR MODESTINO GONÇALVES \\ 
		3.00 & 1.37 & VIRGEM DA LAPA \\ 
		2.00 & 0.91 & Bandeira \\ 
		2.00 & 0.91 & CACHOEIRA DE PAJEÚ \\ 
		2.00 & 0.91 & CARBONITA \\ 
		2.00 & 0.91 & CORONEL MURTA \\ 
		2.00 & 0.91 & CURRAL DE DENTRO \\ 
		2.00 & 0.91 & Felisburgo \\ 
		2.00 & 0.91 & FRANCISCO BADARÓ \\ 
		2.00 & 0.91 & ICARAÍ DE MINAS \\ 
		2.00 & 0.91 & JACINTO \\ 
		2.00 & 0.91 & JURAMENTO \\ 
		2.00 & 0.91 & MONJOLOS \\ 
		2.00 & 0.91 & PADRE PARAÍSO \\ 
		2.00 & 0.91 & PALMÓPOLIS \\ 
		2.00 & 0.91 & PATIS \\ 
		2.00 & 0.91 & PONTO CHIQUE \\ 
		2.00 & 0.91 & PONTO DOS VOLANTES \\ 
		2.00 & 0.91 & RIACHINHO \\ 
		2.00 & 0.91 & RIO DO PRADO \\ 
		2.00 & 0.91 & RUBIM \\ 
		2.00 & 0.91 & SÃO JOÃO DO PARAÍSO \\ 
		2.00 & 0.91 & VARGEM GRANDE DO RIO PARDO \\ 
		2.00 & 0.91 & VÁRZEA DA PALMA \\ 
		1.00 & 0.46 & ARICANDUVA \\ 
		1.00 & 0.46 & BANDEIRA \\ 
		1.00 & 0.46 & DIVISA ALEGRE \\ 
		1.00 & 0.46 & ITAOBIM \\ 
		\hline
		\hline
		\label{semoutorga}
	\end{longtable}
	\end{scriptsize}	
	
	
	\section{Principais propostas}
	
	Apresentamos na tabela \ref{propostas} as principais propostas de resolução elencadas pelos órgãos. As propostas foram agrupadas de forma automatizada visando uma maior aglutinação dos dados, o que permite uma melhor análise qualitativa.
	
	A princípio é possível identificar a \textbf{instalação de clorador de pastilhas} como uma ação considerada prioritária para os órgãos. A \textbf{regularização ambiental} bem como a \textbf{energização do local} aparecem ainda como soluções mais citadas.
	
	\begin{scriptsize}
		% latex table generated in R 3.3.3 by xtable 1.8-2 package
		% Wed Apr 12 14:26:18 2017
		\begin{longtable}{rrp{11cm}}
			\caption{Principais propostas apresentadas pelos Órgãos} \\ 
			\hline
			Frequência & Percentual & Município \\ 
			\hline
			119.00 & 13.28 & Instalar clorador de pastilhas \\ 
			119.00 & 13.28 & realizar regularização ambiental-Outorga \\ 
			97.00 & 10.83 & realizar energização dos poços.Concluir a fase de análise de do índicativo de qualidade de água. \\ 
			67.00 & 7.48 & Instalar clorador de pastilha \\ 
			26.00 & 2.90 & Realizar a regularização ambiental e realizar a energização dos poços \\ 
			25.00 & 2.79 & Apoio da Emater na seleção das comuniaddes a serem selecionadas \\ 
			24.00 & 2.68 & Necessidade de apresentar análise de água \\ 
			24.00 & 2.68 & verificando se há necessidade de instalar o clorador de pastilhas \\ 
			23.00 & 2.57 & tratamento e corrigir ligações prediais \\ 
			23.00 & 2.57 & urbanização \\ 
			20.00 & 2.23 & concluir a fase de análise de do índicativo de qualidade de água. \\ 
			20.00 & 2.23 & Realizar a regularização ambiental e realizar a energização dos poços que ainda não foram energizados \\ 
			13.00 & 1.45 & Bomba \\ 
			12.00 & 1.34 & realizar energização dos poços.Concluir a fase de análise de do índicativo de qualidade de água.Concluir a fase de análise de do índicativo de qualidade de água.Concluir a fase de análise de do índicativo de qualidade de água. \\ 
			11.00 & 1.23 & energia \\ 
			11.00 & 1.23 & Realizar a regularização ambiental -  outorga -  e realizar a energização dos poços. \\ 
			10.00 & 1.12 & realizar energização dos poços. \\ 
			9.00 & 1.00 &  Instalar clorador de pastilha \\ 
			9.00 & 1.00 & concluir análise de do índicativo de qualidade de água. \\ 
			9.00 & 1.00 & tubos \\ 
			8.00 & 0.89 & Concluir a fase de análise de do índicativo de qualidade de água \\ 
			8.00 & 0.89 & Concluir a fase de análise do índicativo de qualidade de água \\ 
			8.00 & 0.89 & Realizar a regularização ambiental e realizar a energização dos poços. \\ 
			8.00 & 0.89 & regularização ambiental e energizar os poços. \\ 
			8.00 & 0.89 & regularização ambiental e energizar os poços.Concluir a fase de análise de do índicativo de qualidade de água. \\ 
			8.00 & 0.89 & Tubos \\ 
			7.00 & 0.78 & bomba \\ 
			7.00 & 0.78 & Equipamentos:Reservatório \\ 
			7.00 & 0.78 & Reservatório \\ 
			6.00 & 0.67 & tubulação \\ 
			5.00 & 0.56 & E Verivifar junto a empresa executora a vazão explorada \\ 
			5.00 & 0.56 & Os poços energizados deve verificar o indicie de qualidade de água. \\ 
			5.00 & 0.56 & regularização ambiental e energizar os poços das comunidades que ainda estão pendentes.Concluir a fase de análise de do índicativo de qualidade de água. \\ 
			5.00 & 0.56 & reservatório \\
			\hline
			\hline
			\label{propostas}
		\end{longtable}
	\end{scriptsize}


	\section{Descrição geral das ações}
	
	Seguimos apresentando uma descrição geral da situação das ações empreendidas por SEDINOR, SECIR e IGAM. Esta descrição possui mais categorias e possibilita, entre outras coisas, a análise de ações cuja tubulação já foi entregue e ações cuja tubulação não foi entregue. Podemos verificar na tabela \ref{desc-geral} e na figura \ref{desc-geral-fig} a distribuição da variável por situação descrita.
	
	\begin{scriptsize}		
	% latex table generated in R 3.3.3 by xtable 1.8-2 package
	% Mon Apr 24 16:23:41 2017
	\begin{longtable}{rrrr}
		\caption{Descrição geral} \\ 
		\hline
		& Frequência & Percentual & \% Válido \\ 
		\hline
		Água imprópria para consumo humano & 6.00 & 1.22 & 1.29 \\ 
		Complementação das ações previstas no Programa ProÁgua & 36.00 & 7.30 & 7.74 \\ 
		Fase: Estudo da análise da água & 207.00 & 41.99 & 44.52 \\ 
		Tubulação e reservatório entregue & 185.00 & 37.53 & 39.78 \\ 
		Tubulação e reservatório não entregue & 31.00 & 6.29 & 6.67 \\ 
		NA's & 28.00 & 5.68 &  \\ 
		Total & 493.00 & 100.00 & 100.00 \\ 
		\hline
		\hline
		\label{desc-geral}
	\end{longtable}
	\end{scriptsize}

	\begin{figure}[!h]
		\centering
		\caption{Descrição Geral}
		\label{desc-geral-fig}
		\includegraphics[scale=1]{descricao_geral.png}
	\end{figure}
	
	\newpage
	
	\thispagestyle{empty}
	\begin{tikzpicture}[remember picture, overlay]
	\node[inner sep=0pt] at (current page.center) {%
		\includegraphics[width=\paperwidth,height=\paperheight]{Contracapa_rastreia.png}%
	};
	\end{tikzpicture}
	\cleardoublepage{}
	
\end{document}
